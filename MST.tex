\documentclass{article}
\usepackage[T2A]{fontenc}
\usepackage{amsmath}
\usepackage{amssymb}
\usepackage{slashbox}
\usepackage{tikz}
\usepackage{pgfplots}
\usepackage{pgfplotstable}
\pgfplotsset{compat = newest}

\title{Анализ генетического алгоритма (1 + (лямбда, лямбда)) на задаче минимального остовного дерева}
\date{2021-04-01}
\author{Антипов Д.С., Шныткин М.П., matveyshnytkin@gmail.com}

\begin{document}
  \pagenumbering{gobble}
  \maketitle
  \newpage
  \pagenumbering{arabic}
  

  \section{Черновик}
  
Рассмотрим фазу остовного дерева.

  \subsection{Оценки сверху}

Зафиксируем $l$.

  \subsubsection{Вероятность 2-флип-мутанта}

Найдём вероятность того, что на фазе мутации произошла конкретная смена одного ребра на другое, при этом остальные $l - 2$ флипов только добавили рёбер к дереву:

$$
P_{2-flip,1} = 
\frac{\binom{m - n}{l - 2}}{\binom{m}{l}} = 
\frac{(m - n)! l! (m - l)!}{(l - 2)! (m - n - l + 2)! m!} =
\frac{(m - n)! l (l - 1) (m - l)!}{(m - n - l + 2)! m!} \geq
$$
$$
\frac{(m - n - l + 3)^{l - 2} l (l - 1)}{m^l} =
\frac{l (l - 1)}{m^2} (\frac{m - n - l + 3}{m})^{l - 2} =
$$
$$
\frac{l (l - 1)}{m^2} (1 - \frac{n + l - 3}{m})^{l - 2}
$$

Т.к. $\forall m \geq 2 : (1 - 1/m)^m \geq 1/4$, то \underline{при $m \geq 2(n + l - 3)$  и $m \geq (l - 2)(n + l - 3)$} выполняется:

$$
P_{2-flip,1} \geq
\frac{l (l - 1)}{m^2} (1 - \frac{n + l - 3}{m})^{l - 2} \geq
\frac{l (l - 1)}{m^2} \frac{1}{4}
$$

  \subsubsection{Вероятность плохого мутанта}

Теперь найдём вероятность мутанта, который будет хуже 2-флип-мутанта и не пройдёт отбор (все $l$ флипов добавляют рёбра):

$$
P_{bm,1} = 
\frac{\binom{m - n + 1}{l}}{\binom{m}{l}} = 
\frac{(m - n + 1)! l! (m - l)!}{l! (m - n + 1 - l)! m!} =
\frac{(m - n + 1)! (m - l)!}{(m - n + 1 - l)! m!} \geq
$$
$$
\frac{(m - n + 2 - l)^l}{m^l} = 
(\frac{m - n + 2 - l}{m})^l =   
(1 - \frac{n + l - 2}{m})^l 
$$

Как уже упоминалось выше $\forall m \geq 2 : (1 - 1/m)^m \geq 1/4$, и \\  \underline{при $m \geq 2(n + l - 2)$  и $m \geq l (n + l - 2)$} выполняется:

$$
P_{bm,1} \geq
(1 - \frac{n + l - 2}{m})^l \geq 
\frac{1}{4}
$$

  \subsubsection{Вероятность поправить ошибки}



  \subsection{Оценка снизу}

Рассмотрим графы с единственным минимальным остовным деревом. В качестве потенциала $X$ выберем число рёбер из MST, не содержащихся в текущем дереве. Рассмотрим последний шаг с $X = 1$, в нём требуется сделать один конкретный флип, чтобы завершить дерево:

$$
E[x_t - x_{t+1} | X_t = 1] = \sum_{l=0}^{m} p_l \sum_{\forall \binom{m - 2}{l - 2} err} p_{flip, err, best} p_{cross, err}
$$

Где $p_{flip, err, best}$ - вероятность того, что после этапа мутации нужный флип произошёл при фиксированных $l - 2$ ошибочных изменений.

И где $p_{cross, err}$ - вероятность удачного кроссовера, в котором флип остался, но все $l - 2$ ошибочных изменения стёрлись.

Мы сразу можем сузить область $l$:

$$
E[x_t - x_{t+1} | X_t = 1] = \sum_{l=2}^{m} p_l \sum_{\forall \binom{m - 2}{l - 2} err} p_{flip, err, best} p_{cross, err}
$$

Т.к. при меньших $l$ флипа одного ребра на другое произойти не может и $p_{flip, err, best} = 0$.


  \subsubsection{Оценка $p_{cross, err}$}

Заметим, что $p_{cross, err}$ не зависит от конкретного распределения ошибок, но зависит только от значения $l - 2$:

$$
p_{cross, err} = p_{cross, l} = 
1 - (1 - \frac{1}{\lambda^2}(1 - \frac{1}{\lambda})^{l - 2})^\lambda \leq
1 - (1 - \frac{1}{\lambda^2}(\frac{1}{e})^\frac{l - 2}{\lambda})^\lambda \leq
$$
$$
1 - (1 - \frac{1}{\lambda}(\frac{1}{e})^\frac{l - 2}{\lambda}) =
\frac{1}{\lambda}(\frac{1}{e})^\frac{l - 2}{\lambda}
$$

Первый переход сделан для $\lambda \geq 2$ используя $\forall m \geq 2$  $\frac{1}{e} \geq (1 - \frac{1}{m})^m$.

Второй переход - неравенство Бернулли ${(1+x)^r \geq 1 + rx}$.

Итог: $p_{cross, err} \leq \frac{1}{\lambda}(\frac{1}{e})^\frac{l - 2}{\lambda}$.


  \subsubsection{Оценка суммы $p_{flip, err, best}$}

Для фиксированного $l$ оценим

$$
\sum_{\forall \binom{m - 2}{l - 2} err} p_{flip, err, best}
$$

Сделаем оценки:

$$
p_{flip, err, best} \leq p_{flip, err, \lambda} = p_{flip, l,  \lambda}
$$

Где $p_{flip, err, \lambda}$ - вероятность флипа с ошибками появиться в качестве одного из мутантов. При таком условии мы не учитываем то, что в выборке могут быть кандидаты лучше нашего, следовательно, $p_{flip, err, \lambda} = p_{flip, l,  \lambda}$.

В итоге, учитывая то, что $p_{flip, err}$ - вероятность появления мутанта c флипом и конкретной ошибкой = $\frac{(l - 1)l}{(m - 1)m}\binom{m - 2}{l - 2}^{-1}$:

$$
\sum_{\forall \binom{m - 2}{l - 2} err} p_{flip, err, best} \leq 
\sum_{\forall \binom{m - 2}{l - 2} err} p_{flip, l,  \lambda} =
$$
$$
\sum_{\forall \binom{m - 2}{l - 2} err} 1 - (1 - \frac{(l - 1)l}{(m - 1)m}\binom{m - 2}{l - 2}^{-1})^\lambda \leq
\sum_{\forall \binom{m - 2}{l - 2} err} 1 - (1 - \lambda\frac{(l - 1)l}{(m - 1)m}\binom{m - 2}{l - 2}^{-1}) \leq
$$
$$
\sum_{\forall \binom{m - 2}{l - 2} err} \lambda\frac{(l - 1)l}{(m - 1)m}\binom{m - 2}{l - 2}^{-1}  = 
\lambda\frac{(l - 1)l}{(m - 1)m}
$$

  \subsubsection{Getting it all together}

Оценим сверху 

$$
E[x_t - x_{t+1} | X_t = 1] = \sum_{l=0}^{m} p_l \sum_{\forall \binom{m - 2}{l - 2} err} p_{flip, err, best} p_{cross, err}
$$

Как мы уже знаем, $p_{cross, err}$ не зависит от err, но зависит только от l, следовательно, вынесем её за сумму:


$$
E[x_t - x_{t+1} | X_t = 1] = \sum_{l=0}^{m} p_l \sum_{\forall \binom{m - 2}{l - 2} err} p_{flip, err, best} p_{cross, err} \leq
$$
$$
\sum_{l=0}^{m} p_l \sum_{\forall \binom{m - 2}{l - 2} err} p_{flip, err, best} \frac{1}{\lambda}(\frac{1}{e})^\frac{l - 2}{\lambda} \leq
$$
$$
\sum_{l=0}^{m} p_l \frac{1}{\lambda}(\frac{1}{e})^\frac{l - 2}{\lambda} \sum_{\forall \binom{m - 2}{l - 2} err} p_{flip, err, best} \leq
$$
$$
\sum_{l=0}^{m} p_l \frac{1}{\lambda}(\frac{1}{e})^\frac{l - 2}{\lambda} \lambda\frac{(l - 1)l}{(m - 1)m} =
$$
$$
\frac{1}{(m - 1)m}\sum_{l=0}^{m} p_l (\frac{1}{e})^\frac{l - 2}{\lambda} (l - 1)l
$$

Для константного $\lambda$ функция $f(l) = (\frac{1}{e})^\frac{l - 2}{\lambda} (l - 1)l$ ограничена сверху константой $C$ (если верить вольфраму и здравому математическому смыслу). Сделаем последний шаг:

$$
E[x_t - x_{t+1} | X_t = 1] \leq \frac{1}{(m - 1)m}\sum_{l=0}^{m} p_l (\frac{1}{e})^\frac{l - 2}{\lambda} (l - 1)l \leq
C(\lambda) \frac{1}{(m - 1)m}\sum_{l=0}^{m} p_l = 
$$
$$
C(\lambda) \frac{1}{(m - 1)m}
$$

\textbf{Тогда из аддитивного дрифта:}

$$
E[T_1] \geq C(\lambda)^{-1} (m - 1)m
$$

\textbf{Для константного  $\lambda > 1$, для графов с единственным MST.}

ТОЧНЕЕ:

$$
E[T_1] \geq \frac{e m^2}{\lambda^2}
$$

\end{document}
